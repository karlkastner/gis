\documentclass[11pt,twoside,a4paper]{article}
%{book}

% This is an automatically generated file.
% Do not edit it.
% Changes to this file are not preserved!

\usepackage{tocloft}
\usepackage{hyperref}
\usepackage{listings}
\lstset{
basicstyle=\small\ttfamily,
columns=flexible,
breaklines=true
}
\setlength{\cftsubsecnumwidth}{3.5em}

\title{Manual for Package:
gis\protect\\Revision 5M
}
\author{Karl K\"astner}
%\date{}

\begin{document}

\maketitle

\tableofcontents

% licence
% abstract


\section{@GeoImg}
\subsection{GeoImg}
${}$
\begin{lstlisting}
 loading, manipulation and writing of geospacial images with world file

\end{lstlisting}
\subsection{crop}
${}$
\begin{lstlisting}
 cropt the image and modify the pgw accordingly

\end{lstlisting}
\subsection{read}
${}$
\begin{lstlisting}
 read geospatial image and pgw from disk and

\end{lstlisting}
\subsection{write}
${}$
\begin{lstlisting}
 write geospatial image and pgw to disk

\end{lstlisting}
\section{gis}
\subsection{GPX}
${}$
\begin{lstlisting}

\end{lstlisting}
\subsection{batavia\_zero}
${}$
\begin{lstlisting}

\end{lstlisting}
\section{centreline/@Centreline}
\subsection{Centreline}
${}$
\begin{lstlisting}

\end{lstlisting}
\subsection{channel\_planimetry}
${}$
\begin{lstlisting}

\end{lstlisting}
\subsection{clip}
${}$
\begin{lstlisting}

\end{lstlisting}
\subsection{connect\_graph}
${}$
\begin{lstlisting}

\end{lstlisting}
\subsection{curvature}
${}$
\begin{lstlisting}

\end{lstlisting}
\subsection{cut}
${}$
\begin{lstlisting}

\end{lstlisting}
\subsection{determine\_width}
${}$
\begin{lstlisting}

\end{lstlisting}
\subsection{distance}
${}$
\begin{lstlisting}

\end{lstlisting}
\subsection{export\_cross\_section}
${}$
\begin{lstlisting}

\end{lstlisting}
\subsection{export\_node}
${}$
\begin{lstlisting}

\end{lstlisting}
\subsection{export\_shp}
${}$
\begin{lstlisting}

\end{lstlisting}
\subsection{find\_nearest\_segment}
${}$
\begin{lstlisting}

\end{lstlisting}
\subsection{from\_polygon}
${}$
\begin{lstlisting}

\end{lstlisting}
\subsection{from\_shp}
${}$
\begin{lstlisting}

\end{lstlisting}
\subsection{get}
${}$
\begin{lstlisting}

\end{lstlisting}
\subsection{init}
${}$
\begin{lstlisting}
obj.seg_S(id(end)) = NaN;

\end{lstlisting}
\subsection{init\_connect}
${}$
\begin{lstlisting}

\end{lstlisting}
\subsection{init\_node\_D}
${}$
\begin{lstlisting}

\end{lstlisting}
\subsection{link\_centreline}
${}$
\begin{lstlisting}

\end{lstlisting}
\subsection{plot}
${}$
\begin{lstlisting}

\end{lstlisting}
\subsection{plot\_connection}
${}$
\begin{lstlisting}

\end{lstlisting}
\subsection{prune}
${}$
\begin{lstlisting}

\end{lstlisting}
\subsection{prune\_leaves}
${}$
\begin{lstlisting}

\end{lstlisting}
\subsection{prune\_manually}
${}$
\begin{lstlisting}

\end{lstlisting}
\subsection{reachable}
${}$
\begin{lstlisting}

\end{lstlisting}
\subsection{remove\_duplicate\_points}
${}$
\begin{lstlisting}

\end{lstlisting}
\subsection{resample}
${}$
\begin{lstlisting}

\end{lstlisting}
\subsection{routing}
${}$
\begin{lstlisting}

\end{lstlisting}
\subsection{routing2}
${}$
\begin{lstlisting}

\end{lstlisting}
\subsection{shp\_resample\_simple}
${}$
\begin{lstlisting}

\end{lstlisting}
\subsection{snmesh}
${}$
\begin{lstlisting}

\end{lstlisting}
\subsection{squeeze}
${}$
\begin{lstlisting}

\end{lstlisting}
\subsection{trim\_ends}
${}$
\begin{lstlisting}

\end{lstlisting}
\subsection{weighed\_connection\_matrix}
${}$
\begin{lstlisting}

\end{lstlisting}
\subsection{xy2sn}
${}$
\begin{lstlisting}

\end{lstlisting}
\section{centreline/@Segment}
\subsection{Segment}
${}$
\begin{lstlisting}

\end{lstlisting}
\subsection{build\_inverse\_index}
${}$
\begin{lstlisting}

\end{lstlisting}
\subsection{connectivity\_matrix}
${}$
\begin{lstlisting}

\end{lstlisting}
\subsection{init\_seg\_id}
${}$
\begin{lstlisting}

\end{lstlisting}
\section{centreline}
\subsection{sn2xy\_quadratic}
${}$
\begin{lstlisting}

\end{lstlisting}
\subsection{thalweg}
${}$
\begin{lstlisting}

\end{lstlisting}
\subsection{xy2sn\_quadratic}
${}$
\begin{lstlisting}

\end{lstlisting}
\section{gis}
\subsection{gpx\_export\_csv}
${}$
\begin{lstlisting}

\end{lstlisting}
\subsection{hgt\_plot}
${}$
\begin{lstlisting}

\end{lstlisting}
\subsection{hgt\_read}
${}$
\begin{lstlisting}
% [ floor(mednan(z(kk))) meannan(z(kk)) min(z(kk)) max(z(kk)) ]

\end{lstlisting}
\subsection{hgt\_read\_all}
${}$
\begin{lstlisting}

\end{lstlisting}
\subsection{hgt\_resample}
${}$
\begin{lstlisting}

\end{lstlisting}
\subsection{nmeatime}
${}$
\begin{lstlisting}

\end{lstlisting}
\subsection{read\_xyz}
${}$
\begin{lstlisting}

\end{lstlisting}
\section{shapefile/@Shp}
\subsection{Shp}
${}$
\begin{lstlisting}
 shape file processing

\end{lstlisting}
\subsection{area}
${}$
\begin{lstlisting}
 area of  polygon shapes

\end{lstlisting}
\subsection{buffer}
${}$
\begin{lstlisting}
 buffer or shrink a polygon by a fixed distance

\end{lstlisting}
\subsection{centroid}
${}$
\begin{lstlisting}

\end{lstlisting}
\subsection{clip}
${}$
\begin{lstlisting}
 crop input shape file to specified polygon

\end{lstlisting}
\subsection{clip\_rect}
${}$
\begin{lstlisting}
 rectrangular crop of the shapefile

\end{lstlisting}
\subsection{close\_polygon}
${}$
\begin{lstlisting}
 close polygon, i.e. make the first point identical to the last

\end{lstlisting}
\subsection{concat}
${}$
\begin{lstlisting}
 concatenate two shapefiles

\end{lstlisting}
\subsection{connect\_network}
${}$
\begin{lstlisting}

 TODO make unique
 attach segments to 
XY = [cvec(shp.X),shp.;
 knnsearch for nearest n neighbours
 for each segment

\end{lstlisting}
\subsection{contour}
${}$
\begin{lstlisting}

\end{lstlisting}
\subsection{copy\_attribute}
${}$
\begin{lstlisting}
 copy attributes from one shapefile to the other

\end{lstlisting}
\subsection{cp}
${}$
\begin{lstlisting}
 copy a shapefile on disk

\end{lstlisting}
\subsection{create}
${}$
\begin{lstlisting}
 create a new shapefile with given geometry

\end{lstlisting}
\subsection{curvature}
${}$
\begin{lstlisting}
 curvature of line segments

\end{lstlisting}
\subsection{cut}
${}$
\begin{lstlisting}


\end{lstlisting}
\subsection{diameter}
${}$
\begin{lstlisting}
 determine diameter of polygon of every element

\end{lstlisting}
\subsection{edges}
${}$
\begin{lstlisting}
 edges of polygons line loops loops

\end{lstlisting}
\subsection{export\_geo}
${}$
\begin{lstlisting}
 export geometry file undestood by SLIM

\end{lstlisting}
\subsection{export\_gpx}
${}$
\begin{lstlisting}
 export data into a gpx file

\end{lstlisting}
\subsection{export\_gpx\_track}
${}$
\begin{lstlisting}
 export a data into a gpx track file

\end{lstlisting}
\subsection{export\_ldb}
${}$
\begin{lstlisting}
 export Delft3D-4 land-boundary

\end{lstlisting}
\subsection{export\_poly}
${}$
\begin{lstlisting}
 export poly-file understood by SLIM

\end{lstlisting}
\subsection{export\_sdf}
${}$
\begin{lstlisting}

\end{lstlisting}
\subsection{export\_spline}
${}$
\begin{lstlisting}
 export splines (for D3D?)

\end{lstlisting}
\subsection{extract\_coastline}
${}$
\begin{lstlisting}

\end{lstlisting}
\subsection{first\_point}
${}$
\begin{lstlisting}
 extract first point of all shapefile features

\end{lstlisting}
\subsection{flat}
${}$
\begin{lstlisting}

\end{lstlisting}
\subsection{generate\_four\_colour\_index}
${}$
\begin{lstlisting}
 unique colour-indices fpr poligons

\end{lstlisting}
\subsection{generate\_rectangle}
${}$
\begin{lstlisting}
 generate rectangular polygon

\end{lstlisting}
\subsection{import\_geo}
${}$
\begin{lstlisting}

\end{lstlisting}
\subsection{import\_poly}
${}$
\begin{lstlisting}
 import poly file

\end{lstlisting}
\subsection{inpolygon}
${}$
\begin{lstlisting}
 test if point is in any of the polygons

\end{lstlisting}
\subsection{join\_lines}
${}$
\begin{lstlisting}
 join line segments

\end{lstlisting}
\subsection{last\_point}
${}$
\begin{lstlisting}
 return last point of features

\end{lstlisting}
\subsection{latlon2utm}
${}$
\begin{lstlisting}
 convert latitude and longitude to utm

\end{lstlisting}
\subsection{length}
${}$
\begin{lstlisting}
 number of points of each feature

\end{lstlisting}
\subsection{length2}
${}$
\begin{lstlisting}
 length of line segments

\end{lstlisting}
\subsection{line2point}
${}$
\begin{lstlisting}
 convert lines to points

\end{lstlisting}
\subsection{link\_lines}
${}$
\begin{lstlisting}
 link lines with same endpoints

\end{lstlisting}
\subsection{make\_clockwise}
${}$
\begin{lstlisting}
 make polygons clockwise

\end{lstlisting}
\subsection{merge}
${}$
\begin{lstlisting}

\end{lstlisting}
\subsection{merge2}
${}$
\begin{lstlisting}

\end{lstlisting}
\subsection{nan2zero}
${}$
\begin{lstlisting}
 replace not a number values with zeros, for writing to disk
 NAN values are not allowed according to the spec

\end{lstlisting}
\subsection{padd\_nan}
${}$
\begin{lstlisting}
 padd NaN at end of features

\end{lstlisting}
\subsection{plot}
${}$
\begin{lstlisting}
 display the shapefile

\end{lstlisting}
\subsection{points}
${}$
\begin{lstlisting}
 returns points of the features

\end{lstlisting}
\subsection{polygon\_boundary}
${}$
\begin{lstlisting}
 

\end{lstlisting}
\subsection{read}
${}$
\begin{lstlisting}
 read shapefile from file

\end{lstlisting}
\subsection{readZ}
${}$
\begin{lstlisting}
 read shapefile with z-data from file
 this is a workaround, as matlab cannot read files with z-data

\end{lstlisting}
\subsection{reassign\_id}
${}$
\begin{lstlisting}
 assign a unique identifier to each feature

\end{lstlisting}
\subsection{remove\_duplicate\_points}
${}$
\begin{lstlisting}
 remove dubplicate points from features

\end{lstlisting}
\subsection{remove\_leaves}
${}$
\begin{lstlisting}

\end{lstlisting}
\subsection{remove\_nan}
${}$
\begin{lstlisting}
 remove NaN points from features

\end{lstlisting}
\subsection{remove\_polygon\_closure}
${}$
\begin{lstlisting}
 remove last points of polygon if they are identical to the first

\end{lstlisting}
\subsection{remove\_short\_elements}
${}$
\begin{lstlisting}
 remove features with few points

\end{lstlisting}
\subsection{renumber}
${}$
\begin{lstlisting}
 generate a new index

\end{lstlisting}
\subsection{resample}
${}$
\begin{lstlisting}
 resample coordinates

\end{lstlisting}
\subsection{resample\_2}
${}$
\begin{lstlisting}
 resample coordinates

\end{lstlisting}
\subsection{resample\_min}
${}$
\begin{lstlisting}
 resample coordinates

\end{lstlisting}
\subsection{resample\_quick}
${}$
\begin{lstlisting}
 resample coordinates

\end{lstlisting}
\subsection{scale}
${}$
\begin{lstlisting}

\end{lstlisting}
\subsection{segment}
${}$
\begin{lstlisting}
 separate disjoint sections of polygons and lines

\end{lstlisting}
\subsection{select\_for\_refinement}
${}$
\begin{lstlisting}
 select elements for refinement

\end{lstlisting}
\subsection{set\_geometry}
${}$
\begin{lstlisting}
 set feature geometry

\end{lstlisting}
\subsection{set\_resolution}
${}$
\begin{lstlisting}
 set resolution for mesh generation

\end{lstlisting}
\subsection{singlepart\_to\_multipart}
${}$
\begin{lstlisting}
 concatenate line segments (parts) of shp data files into one
 same as single part to multipart in qgis
 returns also indices into the original file

\end{lstlisting}
\subsection{skip}
${}$
\begin{lstlisting}
 quick resampling of features by leaving out points

\end{lstlisting}
\subsection{smooth}
${}$
\begin{lstlisting}
 smooth the features

\end{lstlisting}
\subsection{split\_jump}
${}$
\begin{lstlisting}
 split features where distance between points exceeds a threshold

\end{lstlisting}
\subsection{split\_line}
${}$
\begin{lstlisting}
 split line features into single sements

\end{lstlisting}
\subsection{split\_nan}
${}$
\begin{lstlisting}
 splits shp line and polygons at NaN into two different groups

\end{lstlisting}
\subsection{swap\_hemisphere}
${}$
\begin{lstlisting}
 swap northern and southern hemisphere for UTM coordiantes

\end{lstlisting}
\subsection{translate}
${}$
\begin{lstlisting}
 translate coordinates

\end{lstlisting}
\subsection{write}
${}$
\begin{lstlisting}
 write the shapefile to disk

\end{lstlisting}
\section{shapefile}
\subsection{astar\_multi}
${}$
\begin{lstlisting}

\end{lstlisting}
\subsection{astar\_recursive}
${}$
\begin{lstlisting}
 astar path finding algorithm

\end{lstlisting}
\subsection{edge\_chain}
${}$
\begin{lstlisting}

\end{lstlisting}
\subsection{edge\_from\_bnd}
${}$
\begin{lstlisting}

\end{lstlisting}
\subsection{preload\_shp}
${}$
\begin{lstlisting}

\end{lstlisting}
\subsection{read\_gpx}
${}$
\begin{lstlisting}

\end{lstlisting}
\subsection{shapewrite\_\_}
${}$
\begin{lstlisting}
 Copyright (C) 2014,2015 Philip Nienhuis
 
 This program is free software; you can redistribute it and/or modify it
 under the terms of the GNU General Public License as published by
 the Free Software Foundation; either version 3 of the License, or
 (at your option) any later version.
 
 This program is distributed in the hope that it will be useful,
 but WITHOUT ANY WARRANTY; without even the implied warranty of
 MERCHANTABILITY or FITNESS FOR A PARTICULAR PURPOSE.  See the
 GNU General Public License for more details.
 
 You should have received a copy of the GNU General Public License
 along with this program.  If not, see <http://www.gnu.org/licenses/>.
 -*- texinfo -*- 
 @deftypefn {Function File} {@var{status} =} shapewrite (@var{shpstr}, @var{fname})
 Write contents of map- or geostruct to a GIS shape file.

 @var{shpstr} must be a valid mapstruct or geostruct, a struct array with an
 entry for each shape feature, with fields Geometry, BoundingBox, and X and Y
 (mapstruct) or Lat and Lon (geostruct).  For geostructs, Lat and Lon field
 data will be written as X and Y data.  Field Geometry can have data values
 of 'Point', 'MultiPoint', 'Line', or 'Polygon', all case-insensitive.  For
 each shape feature, field BoundingBox should contain the minimum and maximum
 (X,Y) coordinates in a 2x2 array [minX, minY; maxX, maxY].  The X and Y
 fields should contain X (or Latitude) and Y (or Longitude) coordinates for
 each point or vertex as row vectors; for polylines and polygons vertices of
 each subfeature (if present) should be separated by NaN entries.

 @var{fname} should be a valid shape file name, optionally with a '.shp'
 suffix.

 shapewrite produces 2 or 3 files, i.e. a .shp file (the actual shape file),
 a .shx file (index file), and if @var{shpstr} contained additional fields,
 a .dbf file (dBase type 3) with the contents of those additional fields.

 @var{status} is 1 if the shape file set was written successfully, 0
 otherwise.

 @seealso{shaperead, shapeinfo}
 @end deftypefn
 Author: Philip Nienhuis <prnienhuis@users.sf.net>
 Created: 2014-12-30
 Input validation
 Assess shape variable type (oct or ml/geo ml/map)
 Yep. Find out what type
 Assume it is an Octave-style struct read by shaperead
 Assume it is a Matlab-style mapstruct
 Assume it is a Matlab-style geostruct
 Not a supported struct type
 Check file name
 Later on bname.shx and bname.dbf will be read
 Prepare a few things
 Change Lat/Lon fields into X/Y
 Only now (after input checks) open .shp and .shx files & rewind just to be sure
 Write headers in .shp & .shx (identical). First magic number 9994 + 5 zeros
 In between here = filelength in 16-bit words (single). For .shx it's known
 Next, shp file version
 Shape feature type
 Bounding box. Can be run later for ML type shape structs. Fill with zeros
 Skip to start of first record position
 Write shape features one by one
 Write record start pos to .shx file
 Write record contents
 Point
 Record index number
 Record length (fixed)
 Shape type
 Simply write XY cordinates
 MultiPoint
 Record index number
 Record length
 Shape type
 Bounding box
 Nr of points
 Polyline/-gon
 Record index number
 Prepare multipart polygons
 Augment idx for later on, & this trick eliminates trailing NaN rows
 Record length
 Shape type
 Bounding box
 Number of parts, number of points, part pointers
 Write file length into .shp header
 Close files
 Check for dbfwrite function
 Write rest of attributes
 Attributes + shp data in mapstruct
 Attributes + shp data in geostruct

\end{lstlisting}
\subsection{shapewrite\_man}
${}$
\begin{lstlisting}

\end{lstlisting}
\subsection{shp2geo}
${}$
\begin{lstlisting}

\end{lstlisting}
\subsection{shp2kml}
${}$
\begin{lstlisting}

\end{lstlisting}
\subsection{shp\_plot\_attribute}
${}$
\begin{lstlisting}

\end{lstlisting}
\subsection{split\_section}
${}$
\begin{lstlisting}

\end{lstlisting}
\subsection{write\_polygon}
${}$
\begin{lstlisting}

\end{lstlisting}
\section{gis}
\subsection{shp2csv}
${}$
\begin{lstlisting}

\end{lstlisting}
\subsection{write\_xyz}
${}$
\begin{lstlisting}

\end{lstlisting}
\end{document}
